\section{Related Work}
Before we started our project, we looked up several papers that preform similar projects in the past.
The quality and importance of the found papers has been determined by checking the relevance and the credibility. The most relevant and important paper is discussed first in Section \ref{subsec:relatedWorkDetails}. The relevance is looked at from the technical perspective, for example in \emph{Dealing with big data: The case of Twitter} \cite{sangDealingWithBigData} MapReduce is used, which also has been used for this research. We also looked at the relevance in terms of the topic the paper is about, for example how they used Twitter in comparison to the way it is used in this research. Finally the papers have been checked to see if the paper uses big data methods or more traditional methods to analyse the data used for the research.

\subsection{Details} \label{subsec:relatedWorkDetails}
\emph{Dealing with big data: The case of Twitter} \cite{sangDealingWithBigData} has been chosen as most important because it is describing the way that the Twitter data that we will be using for our research is gathered. The paper describes how the data has been filtered (by language for example), and how it has been processed. The data collected for the research of the paper contains more as two billion Dutch tweets, and is still expanding every day. The paper describes the collecting and storage and three case studies: ``relating word frequency to real-life events, finding words related to a topic, and gathering information about conversations'' \cite{sangDealingWithBigData}. The main result of the research is a website where users can search in a huge Twitter dataset, where the result is visualized with graphs and word clouds.

A second related paper is \emph{Twitter based TV rating system} \cite{souzaTwitterBasedTVRatingSystem}, this papers researches how social media data shows the popularity of a TV show. This data is used for advertisers to choose the right TV show for their product and to get the advertisement for a correct price. The system uses a normal MySQL database to store the data, in this case a MySQL database is sufficient because they only search for hashtags/search words related to a TV show. The system scrapes Wikipedia and IMDB for TV shows and then searches on Twitter with the names of the show and its cast. Then the tweets that are found are classified as the class ``relevant'' or ``viewing'', which mean related, but not viewing and viewing the show respectively.

The paper \emph{MapReduce Functions to Analyze Sentiment Information from Social Big Data} \cite{haMapReduceFunctionsToAnalyzeSentiment} analyzes social media data with the Hadoop FileSystem and MapReduce to determine the sentiment of the data. The implemented system is capable of searching for certain keywords, and then determines the sentiment of the tweets. This helps to understand how a certain term is used, and can be used to do market research. 

Then there is the paper \emph{Analysis of Twitter Usage in London, Paris, and New York City} \cite{adnanAnalysisOfTwitterUsageInLondon} This research has analysed Twitter usage in London, Paris and New York on usernames, gender, ethnicity and location. This way they determined what kind of users use Twitter in these cities, and found the high usage areas in these cities. The paper shows graphs and tables with the number of people of a certain ethnicity. They represent the results in tables and graphs.

The paper \emph{Combined analysis of news and Twitter messages} \cite{duCombinedAnalysisOfNewsAndTwitterMessages} uses Twitter data to see how events of large companies show up on social media. The software tries to analyze the tweets of big companies as Facebook, Microsoft, Google, etc. and tries to gather useful and structured data from them. It for example combines news and Twitter messages to get information about an event, for example the announcement of a new Lumia smartphone by Nokia.

The last related paper is called \emph{Modeling retweeting behavior as a game: comparison to empirical results} \cite{learyModelingRetweetingBehaviorAsAGame}, it analyzes retweeting on Twitter. The paper considers a couple of different cases and tries to find out when retweeting occurs, and how/when the original tweeter reacts. They present their theories in tables and a graph.