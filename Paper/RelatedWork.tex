\section{Related Work}
The quality and importance of the found papers has been determined by checking the relevance and the credibility. The most relevant and important paper is discussed first in Section \ref{subsec:relatedWorkDetails}. The relevance is looked at from the technical perspective, for example in \emph{Dealing with big data: The case of Twitter} \cite{sangDealingWithBigData} MapReduce is used, which also has been used for this research. We also looked at the relevance in terms of the topic the paper is about, for example how they used Twitter in comparison to the way it is used in this research. Finally the papers have been checked to see if the paper uses big data methods or more traditional methods to analyse the data used for the research.

\subsection{Details} \label{subsec:relatedWorkDetails}
\emph{Dealing with big data: The case of Twitter} \cite{sangDealingWithBigData} has been chosen as most important because it is describing the way that the Twitter data that we will be using for our research is gathered. The paper describes how the data has been filtered (by language for example), and how it has been processed. The data collected for the research of the paper contains more as two billion Dutch tweets, and is still expanding every day. The paper describes the collecting and storage and three case studies: ``relating word frequency to real-life events, finding words related to a topic, and gathering information about conversations'' \cite{sangDealingWithBigData}.


TODO
Paper 2 is about using Twitter data to generate a rating about a tv show, it is using big data techniques to collect and process the data. In the paper they look for hashtags of popular shows and calculate a popularity score depending on the usage of these hashtags. They also do some analysis on the users that use a particular hashtag, to get to know something about the audience of the show.

Paper 3 describes the usage of a particular big data technique called MapReduce in order to perform analysis on a large set of social media data. By performing this analysis, it is possible to extract information of people’s opinions from the messages they post on social media. This particular paper uses Twitter as an example.

Paper 4 is about analysing Twitter data in different ways. The authors have analysed tweets from London, Paris and New York on usernames, gender, ethnicity and location. They represent the results in tables and graphs.

Paper 5 presents a different way of analysing large sets of Twitter messages. This method does not involve the use of the Hadoop framework, however it does imply the analysis of a large set of data in order to sketch a result about the subject that is being analyzed.

Paper 6 describes the analysis of retweeting behavior on Twitter modeled as a game. This research combines social media behavior with game theory to create an analysis that compares groups of Twitter followers.