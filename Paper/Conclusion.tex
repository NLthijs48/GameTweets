\section{Conclusion}
From the results, we are able to give an answer to our research question: Does the number of tweets about a game in the release year relate to the number of copies sold in Europe?

It turns out there is no strong connection between the number of tweets sent and a game's sales records in Europe. However, there is an indication that a high number of tweets around a game's release date will result in a high number of copies sold. If a lot of people talk about a game on social media, this might have an impact on the game's sales records.  

Certain games have a large variance in tweet counts around their release, after which the hype seems to die down quickly. This can for example be seen with the FIFA franchise. These games tend to release and then be played a lot by various players, then fall off before the release of the next year's FIFA game. 

Other games tend to have a more constant player base. This can be seen with the game The Elder Scrolls V: Skyrim, where the majority of tweets are sent outside of the game's release date. Around 500 to 1000 tweets are sent about this game every day.

The sentiment of tweets might also have an impact on sales records. The tweets might also be about various flaws or bugs in a game, which will likely result in less copies being sold. Therefore, in order to give a better estimation of the number of copies sold depending on the amount of tweets, it might be useful to perform sentiment analysis on these tweets.